%%% Summaries of documents: 
\documentclass[12pt]{article}
\usepackage{geometry}
\geometry{margin = 2cm}
\newcommand{\enter}{\vspace{0.5cm}}
\begin{document}
	
	\section{Laura Kuskopf (2016) Infrared thermal image analysis for the determination of bulk particle properties in cascading particle curtains, Bachelor of Engineering (Honours) Chemical.}
	
		\subsection{Chapter1: Introduction to particle curtains}
		
		\subsection{Chapter2: Examples of IR and VIS image integration (Literature Review)}
		
		\subsection{Chapter3: Methodologies of developing hot particle curtains.}
		
	
		\subsection{Quotable}
	
			\subsubsection{Introduction}
				\begin{itemize}
					\item 
				\end{itemize}
	
		\subsection{General Ideas}
	
			\subsubsection{Introduction}
				\begin{itemize}
					\item 
				\end{itemize}
	
	
	
	
	
	\section{Afshar, Sepideh (2015) Modelling and infra-red thermal imagery of hot particle curtains. PhD thesis, James Cook University.}
	
	\subsection{Quotable}
	
		\subsubsection{Introduction}
			\begin{itemize}
				\item Particle curtains are defined as a stream of particles falling a fixed distance through a gas or fluid
				phase. 
				
				\item They are very common in industrial drying, particularly in the minerals and food industry.
				\item Typical unit operations in drying
				industry are fluidised beds, spray dryers, flighted rotary dryers (FRD) and solid particle receivers
				(SPR).
				
				\item Flighted rotary dryers are used widely in industry because of their simplicity and their
				ability to handle very large throughputs.
				
				\item Particle curtains are important in flighted rotary dryers.
				
				\item Researchers have found that the properties of the individual particles, such as particle
				temperature and particle size, and the operational characteristics such as flow rates and curtain
				depth, solid volume fraction and rates of heat transfer are important in characterising the behaviour
				of particle curtains.
				
				\item CFD has applied successfully to model particle curtains in isothermal conditions; however, there are relatively few CFD studies of hot
				particle curtains. Furthermore, the use of CFD to approximate bulk curtain behaviour has not been
				described.
				
				\item There are a few examples of the use of image analysis to characterise particle curtains.
				These have shown promise and suggest that infrared thermal imagery might provide good data for
				characterising the thermal properties of particle curtains. However, there are no examples of this
				application to two-phase systems. This illustrates a gap in our understanding.
			\end{itemize}
		 
		
		
	
	\subsection{General Ideas}
	
		\subsubsection{Introduction}
			\begin{itemize}
				\item Drying is governed by two-phase heat and mass transfer processes.
				
				\item Characterising the interaction between gas and particles has spurred the interest of researchers for
				decades.
				
				\item Despite the importance of particle curtains in flighted rotary dryers, a
				comprehensive model that describes the influence of these properties has yet to be developed.
				\item There are a few examples of the use of image analysis to characterise particle curtains.
			\end{itemize}


\section{Andrew LEE, (2008) Modelling the Solids Transport Phenomena Within Flighted Rotary Dryers. PhD Thesis, James Cook University}

	\subsection{Quotable}
	
	\subsubsection{Introduction}
	\begin{itemize}
		\item Flighted rotary dryers are used extensively in a range of industries for the control of
		temperature and moisture content of free flowing, particulate solids such as grains,
		sugar and mineral ores. 
		
		\item Despite the extensive use of rotary dryers in industrial applications for many years, a
		general model for a rotary dryer that is applicable to all dryer geometries and
		operating conditions has yet to be developed.
		
		\item A number of models have been
		developed for specific dryers and operating conditions, however these models are
		generally limited to a small range of conditions.
	\end{itemize}
	
	\subsection{General Ideas}
	
	\subsubsection{Introduction}
	\begin{itemize}
		
		\item Whilst flighted rotary dryers are widely used, their complex solids transport
		behaviour, and the difficulty of separating solids transport and heat and mass transfer
		phenomena within the dryer, has proved to be a significant stumbling block in the
		quest to understand their behaviour. Given the complex behaviour of flighted rotary
		dryers, and the lack of design and control procedures, there is a need for a model for
		flighted rotary dryers.
		
	\end{itemize}

\section{Piotr Hellstein, Mariusz Szwedo, (2016) 3D thermography in non-destructive testing of composite structures, AGH University of Science and Technology, Mickiewicza 30 Ave., 30-059 Krakow, Poland \& MONIT SHM, Lublanska 34, 31-476 Krakow, Poland}

\subsection{Quotable}

\subsubsection{Introduction}
\begin{itemize}
	\item 
\end{itemize}

\subsection{General Ideas}

\subsubsection{Introduction}
\begin{itemize}
	\item 
\end{itemize}






\end{document}


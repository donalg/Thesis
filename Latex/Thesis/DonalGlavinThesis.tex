\documentclass[12pt]{article}
\usepackage{amsmath}
\usepackage{geometry}
\usepackage{graphicx}
\usepackage{amssymb}
\usepackage{mathtools}
\usepackage{enumerate}
\usepackage{verbatim}
\usepackage{hyperref}
\usepackage{color}
\usepackage{tikz}
\usetikzlibrary{shapes.geometric, arrows}

\geometry{margin = 2cm}

%%%%%%%%%%%%%%%%%%%%%%%%%%%%%%%%%%%%%%%%%%%%%%%%%%%%%%%%%%%%%%%%%%%%%%%%%%
%% PERSONAL COMMANDS:
%%%%%%%%%%%%%%%%%%%%%%%%%%%%%%%%%%%%%%%%%%%%%%%%%%%%%%%%%%%%%%%%%%%%%%%%%%
% Creating a new indent function to call if i want to indent something but also removing the current indent preset. 
\newlength\tindent 
\setlength{\tindent}{\parindent} 
\setlength{\parindent}{0pt} 
\renewcommand{\indent}{\hspace*{\tindent}} 

\newcommand{\ts}{\textsuperscript}
\newcommand{\pe}{\vspace{0.3cm}}

\newcommand{\mycite}[1]{\ts{\cite{#1}}}


%%%%%%%%%%%%%%%%%%% Flow Chart Diagrams: 
\tikzstyle{position} = [rectangle, rounded corners, minimum width=3cm, minimum height=1cm,text centered, draw=black, fill=blue!30]
%\tikzstyle{io} = [trapezium, trapezium left angle=70, trapezium right angle=110, minimum width=3cm, minimum height=1cm, text centered, draw=black, fill=blue!30]
%\tikzstyle{process} = [rectangle, minimum width=3cm, minimum height=1cm, text centered, draw=black, fill=orange!30]
%\tikzstyle{decision} = [diamond, minimum width=3cm, minimum height=1cm, text centered, draw=black, fill=green!30]
\tikzstyle{arrow} = [thick,<->,>=stealth]


\begin{document}






%%%%%%%%%%%%%%%%%%%%%%%%%%%%%%%%%%%%%%%%%%%%%%%%%%%%%%%%%%%%%%%%%%%%%%%%%%
%% TITLE PAGE
%%%%%%%%%%%%%%%%%%%%%%%%%%%%%%%%%%%%%%%%%%%%%%%%%%%%%%%%%%%%%%%%%%%%%%%%%%

	\begin{titlepage}
		
	\centering
	{\scshape\LARGE James Cook University \par}
	\vspace{0.5cm}
	
	{\huge\bfseries COLLEGE OF SCIENCE \& ENGINEERING\par}
	\vspace{2cm}

	{\large{EG4011/2} \par}

	\vspace{1.5cm}

	{\large{\textbf{Mechanical Engineering}} \par}

	\vspace{3cm}

	{\large{\textbf{INFRARED THERMAL AND VISUAL IMAGE ANALYSIS FOR THE MODELLING OF PROPERTIES IN CASCADING PARTICLE CURTAINS}}	\par}


	\vspace{2cm} 
	{\large Donal Glavin} \\


	\vfill

	Thesis submitted to the School of Engineering \& Physical Sciences in partial fulfilment of the requirements for the degree of \\ {\Large Bachelor of Engineering with Honours (Mechanical Engineering) - Bachelor of Science (Mathematics)} \\ 

	\vspace{2cm} 

	{\large \today \\}

	\vspace{2cm}

\end{titlepage}	

\pagenumbering{roman} % Change pagenumbering to Roman numberals





\begin{comment}

%%%%%%%%%%%%%%%%%%%%%%%%%%%%%%%%%%%%%%%%%%%%%%%%%%%%%%%%%%%%%%%%%%%%%%%%%%
%% STATEMENT OF ACCESS
%%%%%%%%%%%%%%%%%%%%%%%%%%%%%%%%%%%%%%%%%%%%%%%%%%%%%%%%%%%%%%%%%%%%%%%%%%
\section*{Statement of Access}
\addcontentsline{toc}{section}{Statement of Access}
I, the undersigned, author of this work, understand that James Cook University may make this thesis available for use within the the University Library and, via the Australian Digital Theses network, for use elsewhere. \\ 
\vspace{2cm}

I understand that, as an unpublished work, a thesis has significant protection under the Copyright Act and I do not wish to place any further restriction on access to this work. \\ 
\vspace{5cm}

\underline{\hspace{4cm}} \hfill \underline{\hspace{3cm} \today}  \\ 
Signature \hfill Date \\


\pagebreak







%%%%%%%%%%%%%%%%%%%%%%%%%%%%%%%%%%%%%%%%%%%%%%%%%%%%%%%%%%%%%%%%%%%%%%%%%%
%% DECLARATION OF SOURCES
%%%%%%%%%%%%%%%%%%%%%%%%%%%%%%%%%%%%%%%%%%%%%%%%%%%%%%%%%%%%%%%%%%%%%%%%%%
\section*{Declaration of Sources}
\addcontentsline{toc}{section}{Declaration of Sources}
I declare that this thesis is my own work and has not been submitted in any form for another degree or diploma at any university or other institution of tertiary education. Information derived from the published of unpublished work of others has been acknowledged in this text and a list of references id given. 
\vspace{5cm}

\underline{\hspace{4cm}} \hfill \underline{\hspace{3cm} \today}  \\ 
Signature \hfill Date \\


\pagebreak






%%%%%%%%%%%%%%%%%%%%%%%%%%%%%%%%%%%%%%%%%%%%%%%%%%%%%%%%%%%%%%%%%%%%%%%%%%
%% ABSTRACT
%%%%%%%%%%%%%%%%%%%%%%%%%%%%%%%%%%%%%%%%%%%%%%%%%%%%%%%%%%%%%%%%%%%%%%%%%%
\section*{Abstract}
\addcontentsline{toc}{section}{Abstract}

{\color{red} To be completed at later date.} 
\pagebreak





%%%%%%%%%%%%%%%%%%%%%%%%%%%%%%%%%%%%%%%%%%%%%%%%%%%%%%%%%%%%%%%%%%%%%%%%%%
%% ACKNOWLEDGEMENTS
%%%%%%%%%%%%%%%%%%%%%%%%%%%%%%%%%%%%%%%%%%%%%%%%%%%%%%%%%%%%%%%%%%%%%%%%%%
\section*{Acknowledgements}
\addcontentsline{toc}{section}{Acknowledgements}

{\color{red} To be completed at later date.} 





\end{comment}

%%%%%%%%%%%%%%%%%%%%%%%%%%%%%%%%%%%%%%%%%%%%%%%%%%%%%%%%%%%%%%%%%%%%%%%%%%
%% CONTENTS PAGE
%%%%%%%%%%%%%%%%%%%%%%%%%%%%%%%%%%%%%%%%%%%%%%%%%%%%%%%%%%%%%%%%%%%%%%%%%%
\pagebreak
\tableofcontents
\pagebreak

%%%%%%%%%%%%%%%%%%%%%%%%%%%%%%%%%%%%%%%%%%%%%%%%%%%%%%%%%%%%%%%%%%%%%%%%%%
%% LISTS OF FIGURES./Bibliography/
%%%%%%%%%%%%%%%%%%%%%%%%%%%%%%%%%%%%%%%%%%%%%%%%%%%%%%%%%%%%%%%%%%%%%%%%%%
%\pagebreak
%\listoffigures






%%%%%%%%%%%%%%%%%%%%%%%%%%%%%%%%%%%%%%%%%%%%%%%%%%%%%%%%%%%%%%%%%%%%%%%%%%
%% LISTS OF TERMS
%%%%%%%%%%%%%%%%%%%%%%%%%%%%%%%%%%%%%%%%%%%%%%%%%%%%%%%%%%%%%%%%%%%%%%%%%%
%\pagebreak



\pagenumbering{arabic} % Change page numbering to Numerals







%%%%%%%%%%%%%%%%%%%%%%%%%%%%%%%%%%%%%%%%%%%%%%%%%%%%%%%%%%%%%%%%%%%%%%%%%%
%%%%%%%%%%%%%%%%%%%%%%%%%%%%%%%%%%%%%%%%%%%%%%%%%%%%%%%%%%%%%%%%%%%%%%%%%%
%%%%%%%%%%%%%%%%%%%%%%%%%%%%%%%%%%%%%%%%%%%%%%%%%%%%%%%%%%%%%%%%%%%%%%%%%%
%%%%%%%%%%%%%%%%%%%%%%%%%%%%%%%%%%%%%%%%%%%%%%%%%%%%%%%%%%%%%%%%%%%%%%%%%%
%% START OF DOCUMENT:	
%%%%%%%%%%%%%%%%%%%%%%%%%%%%%%%%%%%%%%%%%%%%%%%%%%%%%%%%%%%%%%%%%%%%%%%%%%
%%%%%%%%%%%%%%%%%%%%%%%%%%%%%%%%%%%%%%%%%%%%%%%%%%%%%%%%%%%%%%%%%%%%%%%%%%
%%%%%%%%%%%%%%%%%%%%%%%%%%%%%%%%%%%%%%%%%%%%%%%%%%%%%%%%%%%%%%%%%%%%%%%%%%
%%%%%%%%%%%%%%%%%%%%%%%%%%%%%%%%%%%%%%%%%%%%%%%%%%%%%%%%%%%%%%%%%%%%%%%%%%









%%%%%%%%%%%%%%%%%%%%%%%%%%%%%%%%%%%%%%%%%%%%%%%%%%%%%%%%%%%%%%%%%%%%%%%%%%
%% INTRODUCTION
%%%%%%%%%%%%%%%%%%%%%%%%%%%%%%%%%%%%%%%%%%%%%%%%%%%%%%%%%%%%%%%%%%%%%%%%%%
\section{Introduction}

% Introduce the general purpose:
%%%%%%%%%%%%%%%%%%%%%%%%%%%%%%%%%%%%%%%%%%%%%%%%%%%%%%%%%%%%%%%%%%%%%%%%%%
Understanding temperature distributions in Engineering applications plays an invaluable role toward understanding the effectiveness of design. 
As one of the most frequently measured physical quantity, the quantification of temperate values and their distributions allows for design goals to be achieved in a far more succinct manner and plays a key role in a diverse range of engineering systems. One particularly challenging engineering process that proves problematic in obtaining thermal data for is the particle curtain. \\

\pe 


% Define particle curtains and Potential uses: 
%%%%%%%%%%%%%%%%%%%%%%%%%%%%%%%%%%%%%%%%%%%%%%%%%%%%%%%%%%%%%%%%%%%%%%%%%%
Particle curtains are defined as a stream of particles falling a fixed distance through a gas or fluid phase \mycite{AfsharCurtainPhd}. They are very common in industrial drying, particularly in the minerals and food industry \mycite{AfsharCurtainPhd}. Furthermore, particle curtains are steadily emerging in promising new renewable energy technologies for use in concentrating solar power (CSP) plants. These solar particle receiver designs are currently in the early demonstration phase, delivering improved thermal efficiency through their direct storage of heat within sand-like particles \ts{\cite{christian2015system, viebahn2011potential}}. With regards to the Industrial and Mineral Industries a common technique for particle drying is the use of Flighted rotary dryers (FRD). Flighted rotary dryers are used extensively in a large range of industries for the control of
temperature and moisture content of free flowing, particulate solids such as grains,
sugar and mineral ores \mycite{AndrewLeePhd}. However, Whilst flighted rotary dryers are widely used, their complex solids transport behaviour, and the difficulty of separating solids transport and heat and mass transfer
phenomena within the dryer, has proven to be a significant issue in regards to understanding their behaviour. Given the complex behaviour of flighted rotary dryers, and the lack of design and control procedures, there is a need for a model for flighted rotary dryers \mycite{AndrewLeePhd}. \\ 

\pe 


\begin{comment}
Particle curtains are defined as a continuous stream of particles made to fall in a curtain-like
shape through a gaseous medium. They are employed in a wide variety of industries as heat
exchangers for particulate mediums, due primarily to their simplicity and low operational costs \ts{\cite{AfsharCurtainPhd, AndrewLeePhd}}. Examples include the flighted rotary dryer (FRD) and the hopper, which are common in mineral and pharmaceutical industries. However, particle curtains are steadily emerging in promising new renewable energy technologies for use in concentrating solar power (CSP) plants. These solar particle receiver designs are currently in the early demonstration phase, delivering improved thermal efficiency through their direct storage of heat within sand-like particles \ts{\cite{christian2015system, viebahn2011potential}}.
\end{comment}


% Introduce current analysis techniques:
%%%%%%%%%%%%%%%%%%%%%%%%%%%%%%%%%%%%%%%%%%%%%%%%%%%%%%%%%%%%%%%%%%%%%%%%%%
Currently there are two major branches of research being conducted on particle curtains. The first is of which relates to computational modelling of particle curtains, one of which is the development of Computational Fluid Dynamics (CFD) based models. CFD has been applied successfully to model particle curtains in isothermal conditions; however, there are relatively few CFD studies of hot
particle curtains. Furthermore, the use of CFD to approximate bulk curtain behaviour has not been
described \mycite{AfsharCurtainPhd}. Another of these methods relates to DEM

% Define gap in these techniques: 
%%%%%%%%%%%%%%%%%%%%%%%%%%%%%%%%%%%%%%%%%%%%%%%%%%%%%%%%%%%%%%%%%%%%%%%%%%


% Introduce Innovations in Methodology and/or Technology that may affect the way we look at these problems:  
%%%%%%%%%%%%%%%%%%%%%%%%%%%%%%%%%%%%%%%%%%%%%%%%%%%%%%%%%%%%%%%%%%%%%%%%%%


% Identify how these innovations may/have affect the feild or similar fields:
%%%%%%%%%%%%%%%%%%%%%%%%%%%%%%%%%%%%%%%%%%%%%%%%%%%%%%%%%%%%%%%%%%%%%%%%%%


% Identify the how their implementations how they can affect this field:
%%%%%%%%%%%%%%%%%%%%%%%%%%%%%%%%%%%%%%%%%%%%%%%%%%%%%%%%%%%%%%%%%%%%%%%%%%



\subsection{Objectives}
% Define the objectives of this these:


%%%%%%%%%%%%%%%%%%%%%%%%%%%%%%%%%%%%%%%%%%%%%%%%%%%%%%%%%%%%%%%%%%%%%%%%%%
%% LITERATURE REVIEW
%%%%%%%%%%%%%%%%%%%%%%%%%%%%%%%%%%%%%%%%%%%%%%%%%%%%%%%%%%%%%%%%%%%%%%%%%%
\section{Literature Review}



\subsection{Introduction}
% Introduce Literature Review

In this Literature Review, a general description of thermodynamic heat transfer will be looked at along with the physical nature of particle curtains and the relevant thermodynamic principles that affect their behaviour, furthermore the relevance of particle curtains in real world applications will be looked at and current models that 

\subsection{Thermodynamic Principles} 
% Introduce general Thermodynamic Prociples


Within the study of thermodynamics the phenomena of heat transfer is defined as thermal energy in transit due to a spatial temperature difference\mycite{bergman2011fundamentals}.
There are several modes in which heat/energy transfer (Q) can be transferred from differential temperature zones. When a temperature gradient exists in a medium or between a two mediums, which may be a solid, fluid or both there are three types of energy transfer that may exist: 
\begin{itemize}
	\item Conduction - Refers to the heat transfer that will occur across or within the medium. 
	\item Convection - Refers to the heat transfer that will occur between a surface and a moving fluid. 
	\item Radiation - Refers to the heat transfer that will occur from any surface at a temperature greater than absolute zero. 
\end{itemize}

Due to the inherently complex behaviour of matter, these three modes of energy transfer are described in very different ways although sharing the common trait to propagate toward equilibrium. A more in depth description of these modes can be seen below:  

\subsubsection{Conduction}
Conduction may be viewed as the transfer of energy from the more energetic to the less
energetic particles of a substance due to interactions between the particles \mycite{bergman2011fundamentals}. This phenomena occurs very regularly in real world scenarios with the rate of heat transfer sharing a direct proportionality between the temperature differential. Fourier's Law mathematically describes this transfer:

\begin{align} \label{condCart}
\frac{\partial q}{\partial t} = -k \left( \frac{\partial T}{\partial x} \hat{i} + \frac{\partial T}{\partial y} \hat{j} + \frac{\partial T}{\partial z} \hat{k} \right)
\end{align}
Where: \\ 
\begin{tabular}{l | l}
	$\frac{\partial q}{\partial t}$ & Is the thermal energy flux $(W/m^2)$ \\ 
	$k$ & Is the thermal transport property (Thermal conductivity) $(W/m \cdot K)$ \\ 
	$\frac{\partial T}{\partial(x,y,z)}$ & Is the temperature gradient in each the $x,y$ and $z$ directions respectively. 
\end{tabular}


\subsubsection{Convection}
The convection heat transfer mode is comprised of two mechanisms. In addition to energy
transfer due to random molecular motion (diffusion), energy is also transferred by the bulk, or macroscopic, motion of the fluid. This fluid motion is associated with the fact that, at any instant, large numbers of molecules are moving collectively or as aggregates.Such motion, in the presence of a temperature gradient, contributes to heat transfer\mycite{bergman2011fundamentals}.  
\pe  

This mode of heat transfer is considered as the most complex of the three due to the inherent chaotic behaviour of fluids. There are two sub categories of convective heat transfer: The first relating to natural convective behaviour where the temperature differentials between the fluid and the surface result in changes in density and thus fluid motion based on the physical properties of buoyancy. The second relates to forced convective heat transfer where there is some relative velocity difference between the bulk of the fluid and the surface of the object, in this instance the formation of a fluid velocity boundary layer occurs over the object base on the no slip condition which in turn provides the medium in which heat transfer occurs in the form of a temperature distribution within this boundary layer. In general regardless of the nature of the convection heat transfer process, the heat transfer is defined by Newton's Law of Cooling as seen in equation \ref{NetonsCooling}: 

\begin{align} \label{NetonsCooling}
	\frac{\partial q}{\partial t} = h(T_s - T_\infty) 
\end{align}
Where: \\ 
\begin{tabular}{l | l}
	$\frac{\partial q}{\partial t}$ & Is the thermal energy flux $(W/m^2)$ \\ 
	$h$ & Is the thermal transport property (Convective heat transfer coefficient) $(W/m^2 \cdot K)$ \\ 
	$T_s$ & Is the temperature of the objects surface \\ 
	$T_\infty$ & Is the bulk fluid temperature \\   
\end{tabular}

lala yeah boy!

\subsubsection{Radiation}

\subsection{Particle Curtains}
% Introduce uses practicle applications and current research standards on Particle Curtains
\subsubsection{Computational Modelling} 

\subsubsection{Experimental Investigations}

\subsection{Infra-red Thermography}
% Introduce Infra-red cameras, How they work, their limitations, and their current uses. 

\subsection{Image Analysis/Processing}
% Introduce current Image analysis and processing techniques along with statistical analysis techniques used in image analysis and processing. 

\subsection{Statistical Methods currently used in Image Processing/Analysis} 


\subsection{Camera Modelling}
% Talk about the Camera model and it's uses in real world scanrios such as control systems 

\subsection{Model building}
% Describe techniques that use Image analysis/processing along with the camera model that allow the development of model building from video footage.


\subsection{Conclusions}

\section{Methodology} 


%%%%%%%%%%%%% TOPICS OF INTERIST (with respect to research) 
% Photogrametry / Videometry
% RBG-D camera 
% Lidar 
% Point mapping 
% Development of camera model

% Thermography
% Thermodynamics of Discrete particles
% CFD analysis of Particle Curtains
% DEM analysis of Particle Curtains
% Particle Curtain Research


\addcontentsline{toc}{section}{References}
\bibliographystyle{ieeetr}
\bibliography{references.bib}

\end{document}
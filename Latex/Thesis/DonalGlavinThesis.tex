\documentclass[12pt]{article}
\usepackage{amsmath}
\usepackage{geometry}
\usepackage{graphicx}
\usepackage{amssymb}
\usepackage{mathtools}
\usepackage{enumerate}
\usepackage{verbatim}
%\usepackage{logreq}
%\usepackage[
%backend=biber, 
%style=alphabetic, 
%citestyle=authoryear
%]{biblatex}
%\usepackage{apacite}

%\addbibresource{references.bib}

\usepackage{color}
\usepackage{tikz}
\usetikzlibrary{shapes.geometric, arrows}

% Creating a new indent function to call if i want to indent 
% something but also removing the current indent preset. 
\newlength\tindent % Creates measurement variable
\setlength{\tindent}{\parindent} % Sets variable to length of standard indent
\setlength{\parindent}{0pt} % Sets standard indent to 0

\renewcommand{\indent}{\hspace*{\tindent}} % creates a new function called indent which executes a space the length of the standard preset indent which is identified by the * symbol to occur at the start of a new line
\newcommand{\ts}{\textsuperscript}
\newcommand{\pe}{\vspace{0.3cm}}

%%%%%%%%%%%%%%%%%%% Flow Chart Diagrams: 
\tikzstyle{position} = [rectangle, rounded corners, minimum width=3cm, minimum height=1cm,text centered, draw=black, fill=blue!30]
%\tikzstyle{io} = [trapezium, trapezium left angle=70, trapezium right angle=110, minimum width=3cm, minimum height=1cm, text centered, draw=black, fill=blue!30]
%\tikzstyle{process} = [rectangle, minimum width=3cm, minimum height=1cm, text centered, draw=black, fill=orange!30]
%\tikzstyle{decision} = [diamond, minimum width=3cm, minimum height=1cm, text centered, draw=black, fill=green!30]
\tikzstyle{arrow} = [thick,<->,>=stealth]

\geometry{margin = 2cm}


\begin{document}

%%%%%%%%%%%%%%%%%%%%%%%%%%%%%%%%%%%%%%%%%%%%%%%%%%%%%%%%%%%%%%%%%%%%%%%%%%
%% TITLE PAGE
%%%%%%%%%%%%%%%%%%%%%%%%%%%%%%%%%%%%%%%%%%%%%%%%%%%%%%%%%%%%%%%%%%%%%%%%%%

	\begin{titlepage}

	\centering
	%\includegraphics[width=0.15\textwidth]{example-image-1x1}\par\vspace{1cm}
	{\scshape\LARGE James Cook University \par}
	%\vspace{1cm}
	%{\scshape\Large Engineering Practice Report \par}
	\vspace{0.5cm}
	{\huge\bfseries COLLEGE OF SCIENCE \& ENGINEERING\par}
	\vspace{2cm}

	{\large{EG4011/2} \par}

	\vspace{1.5cm}

	{\large{\textbf{Mechanical Engineering}} \par}

	\vspace{3cm}

	{\large{\textbf{INFRARED THERMAL AND VISUAL IMAGE ANALYSIS FOR THE MODELLING OF PROPERTIES IN CASCADING PARTICLE CURTAINS}}	\par}


	\vspace{2cm} 
	{\large Donal Glavin} \\


	\vfill

	Thesis submitted to the School of Engineering \& Physical Sciences in partial fulfilment of the requirements for the degree of \\ {\Large Bachelor of Engineering with Honours (Mechanical Engineering) - Bachelor of Science (Mathematics)} \\ 

	\vspace{2cm} 

	\today

	\vspace{2cm}
	% Bottom of the page
	%{\large \today\par}
\end{titlepage}	

\pagenumbering{roman} % Change pagenumbering to Roman numberals

%%%%%%%%%%%%%%%%%%%%%%%%%%%%%%%%%%%%%%%%%%%%%%%%%%%%%%%%%%%%%%%%%%%%%%%%%%
%% STATEMENT OF ACCESS
%%%%%%%%%%%%%%%%%%%%%%%%%%%%%%%%%%%%%%%%%%%%%%%%%%%%%%%%%%%%%%%%%%%%%%%%%%
\section*{Statement of Access}
\addcontentsline{toc}{section}{Statement of Access}
I, the undersigned, author of this work, understand that James Cook University may make this thesis available for use within the the University Library and, via the Australian Digital Theses network, for use elsewhere. \\ 
\vspace{2cm}

I understand that, as an unpublished work, a thesis has significant protection under the Copyright Act and I do not wish to place any further restriction on access to this work. \\ 
\vspace{5cm}


%Signature \hfill Date \\ 

\pagebreak

%%%%%%%%%%%%%%%%%%%%%%%%%%%%%%%%%%%%%%%%%%%%%%%%%%%%%%%%%%%%%%%%%%%%%%%%%%
%% DECLARATION OF SOURCES
%%%%%%%%%%%%%%%%%%%%%%%%%%%%%%%%%%%%%%%%%%%%%%%%%%%%%%%%%%%%%%%%%%%%%%%%%%
\section*{Declaration of Sources}
\addcontentsline{toc}{section}{Declaration of Sources}
I declare that this thesis is my own work and has not been submitted in any form for another degree or diploma at any university or other institution of tertiary education. Information derived from the published of unpublished work of others has been acknowledged in this text and a list of references id given. 

\pagebreak

%%%%%%%%%%%%%%%%%%%%%%%%%%%%%%%%%%%%%%%%%%%%%%%%%%%%%%%%%%%%%%%%%%%%%%%%%%
%% ABSTRACT
%%%%%%%%%%%%%%%%%%%%%%%%%%%%%%%%%%%%%%%%%%%%%%%%%%%%%%%%%%%%%%%%%%%%%%%%%%
\section*{Abstract}
\addcontentsline{toc}{section}{Abstract}

{\color{red} To be completed at later date.} 
\pagebreak



%%%%%%%%%%%%%%%%%%%%%%%%%%%%%%%%%%%%%%%%%%%%%%%%%%%%%%%%%%%%%%%%%%%%%%%%%%
%% ACKNOWLEDGEMENTS
%%%%%%%%%%%%%%%%%%%%%%%%%%%%%%%%%%%%%%%%%%%%%%%%%%%%%%%%%%%%%%%%%%%%%%%%%%
\section*{Acknowledgements}
\addcontentsline{toc}{section}{Acknowledgements}

{\color{red} To be completed at later date.} 


%%%%%%%%%%%%%%%%%%%%%%%%%%%%%%%%%%%%%%%%%%%%%%%%%%%%%%%%%%%%%%%%%%%%%%%%%%
%% CONTENTS PAGE
%%%%%%%%%%%%%%%%%%%%%%%%%%%%%%%%%%%%%%%%%%%%%%%%%%%%%%%%%%%%%%%%%%%%%%%%%%
\pagebreak
\tableofcontents

%%%%%%%%%%%%%%%%%%%%%%%%%%%%%%%%%%%%%%%%%%%%%%%%%%%%%%%%%%%%%%%%%%%%%%%%%%
%% LISTS OF FIGURES./Bibliography/
%%%%%%%%%%%%%%%%%%%%%%%%%%%%%%%%%%%%%%%%%%%%%%%%%%%%%%%%%%%%%%%%%%%%%%%%%%
\pagebreak
\listoffigures


%%%%%%%%%%%%%%%%%%%%%%%%%%%%%%%%%%%%%%%%%%%%%%%%%%%%%%%%%%%%%%%%%%%%%%%%%%
%% LISTS OF TERMS
%%%%%%%%%%%%%%%%%%%%%%%%%%%%%%%%%%%%%%%%%%%%%%%%%%%%%%%%%%%%%%%%%%%%%%%%%%
\pagebreak

\pagebreak

\pagenumbering{arabic} % Change page numbering to Numerals

%%%%%%%%%%%%%%%%%%%%%%%%%%%%%%%%%%%%%%%%%%%%%%%%%%%%%%%%%%%%%%%%%%%%%%%%%%
%%%%%%%%%%%%%%%%%%%%%%%%%%%%%%%%%%%%%%%%%%%%%%%%%%%%%%%%%%%%%%%%%%%%%%%%%%
%%%%%%%%%%%%%%%%%%%%%%%%%%%%%%%%%%%%%%%%%%%%%%%%%%%%%%%%%%%%%%%%%%%%%%%%%%
%%%%%%%%%%%%%%%%%%%%%%%%%%%%%%%%%%%%%%%%%%%%%%%%%%%%%%%%%%%%%%%%%%%%%%%%%%
%% START OF DOCUMENT:	
%%%%%%%%%%%%%%%%%%%%%%%%%%%%%%%%%%%%%%%%%%%%%%%%%%%%%%%%%%%%%%%%%%%%%%%%%%
%%%%%%%%%%%%%%%%%%%%%%%%%%%%%%%%%%%%%%%%%%%%%%%%%%%%%%%%%%%%%%%%%%%%%%%%%%
%%%%%%%%%%%%%%%%%%%%%%%%%%%%%%%%%%%%%%%%%%%%%%%%%%%%%%%%%%%%%%%%%%%%%%%%%%
%%%%%%%%%%%%%%%%%%%%%%%%%%%%%%%%%%%%%%%%%%%%%%%%%%%%%%%%%%%%%%%%%%%%%%%%%%



%%%%%%%%%%%%%%%%%%%%%%%%%%%%%%%%%%%%%%%%%%%%%%%%%%%%%%%%%%%%%%%%%%%%%%%%%%
%% INTRODUCTION
%%%%%%%%%%%%%%%%%%%%%%%%%%%%%%%%%%%%%%%%%%%%%%%%%%%%%%%%%%%%%%%%%%%%%%%%%%
\section{Introduction}

% Introduce the general purpose:
Understanding temperature distributions in Engineering applications plays an invaluable role toward understanding the effectiveness of design. 
As the most frequently measured physical quantity, the quantification of temperate values and their distributions allows for design goals to be achieved in a far more succinct manner and plays a key role in a diverse range of engineering systems. One particularly challenging engineering process that proves problematic in obtaining thermal data for is the particle curtain. \\

\pe 

% Define particle curtains and Potential uses: 
Particle curtains are defined as a continuous stream of particles made to fall in a curtain-like
shape through a gaseous medium. They are employed in a wide variety of industries as heat
exchangers for particulate mediums, due primarily to their simplicity and low operational costs \ts{\cite{AfsharCurtainPhd, AndrewLeePhd}}. Examples include the flighted rotary dryer (FRD) and the hopper, which are common in mineral and pharmaceutical industries. However, particle curtains are steadily emerging in promising new renewable energy technologies for use in concentrating solar power (CSP) plants. These solar particle receiver designs are currently in the early demonstration phase, delivering improved thermal efficiency through their direct storage of heat within sand-like particles \ts{\cite{christian2015system, viebahn2011potential}}.

% Introduce current analysis techniques 


% Define gap in these techniques:


% Introduce Innovations in Methodology and/or Technilogy that may affect the way we look at these problems: 
 
% Identify how these inovations may/have affect the feild or similar fields

% Identify the how their implementations how they can affect this field 

\subsection{Objectives}
% Define the objectives of this these:

















%%%%%%%%%%%%%%%%%%%%%%%%%%%%%%%%%%%%%%%%%%%%%%%%%%%%%%%%%%%%%%%%%%%%%%%%%%
%% LITERATURE REVIEW
%%%%%%%%%%%%%%%%%%%%%%%%%%%%%%%%%%%%%%%%%%%%%%%%%%%%%%%%%%%%%%%%%%%%%%%%%%
\section{Literature Review}




%%%%%%%%%%%%% TOPICS OF INTERIST (with respect to research) 
% Photogrametry / Videometry
% RBG-D camera 
% Lidar 
% Point mapping 
% Development of camera model

% Thermography
% Thermodynamics of Discrete particles
% CFD analysis of Particle Curtains
% DEM analysis of Particle Curtains
% Particle Curtain Research







\addcontentsline{toc}{section}{References}
\bibliographystyle{apalike}
\bibliography{references.bib}

\end{document}